
\documentclass{article}
\usepackage{graphicx}

\begin{document}
\title{ Application Assignment: Filter \#114}

\author{G V V Sharma \\06407010\\ e-mail: gadepall@gmail.com	}

\maketitle
%---------------------------------------------------------------

\begin{enumerate}
\item {\em Un-normalized Discrete-Time Filter Specifications:} 
\begin{itemize}
 \item We are supposed to design the equivalent FIR and IIR filter realizations for filter number 114.  
This is a {\em bandpass} filter.
 
 \item The {\em passband} ($\delta_1$) and {\em stopband} ($\delta_2$) {\em tolerances} are given to be equal, so $\delta_1 = \delta_2 = \delta = 0.15$.
 \item The {\em passband} of filter number $j$, $j$ going from 109 to 135 is from \{3 + 0.6(j-109)\}kHz
to \{3 + 0.6(j-107)\}kHz.  Since our filter number is 114, substituting $j = 114$ gives the {\em passband}
range for our bandpass filter as $6$ kHz - $7.2$ kHz.  Hence, the un-normalized discrete time filter
{\em passband} (natural) frequencies are $F_{p1} = 7.2$ kHz
and $F_{p2} = 6.0$ kHz.
\item The {\em transition band} for bandpass filters is $\Delta F = 0.3$ kHz on either side of the passband.
Hence, the un-normalized {\em stopband} frequencies are $F_{s1} = 7.2 + 0.3 = 7.5$ kHz and $F	_{s2} = 6.0 - 0.3 = 5.7$ kHz.
\item We are supposed to design filters whose {\em stopband} is {\em monotonic} and {\em passband equiripple}.  
Hence, we use the {\em Chebyschev approximation} to design our bandpass IIR filter.
 \end{itemize}
 
 \item {\em Normalized Digital Filter Specifications:}
 
 \begin{itemize}
 
 \item The corresponding {\em normalized} digital filter {\em passband} frequencies are
$\omega_{p1} = 2\pi\frac{F_{p1}}{F_s}  = 0.3\pi$  and $\omega_{p2} = 2\pi\frac{F_{p2}}{F_s}  = 0.25 \pi$ kHz.  The centre frequency is then given by  $\omega_c = \frac{\omega_{p1} + \omega_{p2}}{2} = 0.275\pi$.  

\item The {\em normalized stopband} frequencies are $\omega_{s1} = 0.3125 \pi$  and $\omega_{s2} =  0.2375 \pi$.

\end{itemize}

\item {\em The Analog Filter Specifications Using The Bilinear Transformation:}

\begin{itemize}
\item The {\em analog  passband} frequencies are $\Omega_{p1} = 0.5095$, $\Omega_{p2} = 0.4142$
\item The {\em analog  stopband} frequencies are $\Omega_{s1} = 0.5345$, $\Omega_{s2} = 0.3914$
\end{itemize}

\item {\em The Bandpass-Lowpass Frequency Transformation:}

\begin{itemize}

\item If $\Omega$ be the {\em analog bandpass} frequency and $\Omega_L$ be the corresponding {\em analog lowpass} frequency,
\begin{equation}
\Omega_L = \frac{\Omega^2 - \Omega_0^2}{B\Omega}.
\end{equation}

\item $\Omega_0 = \sqrt{\Omega_{p1}\Omega_{p2}} = 0.4594$ and $B = \Omega_{p1} - \Omega_{p2} = 0.0953$
\end{itemize}

\item {\em The Frequency Transformed Lowpass Analog Filter Specifications:}
The low pass filter has
the {\em passband edge} at $\Omega_{Lp} = 1$ and {\em stopband edges} at $\Omega_{Ls_1} = 1.4653$ and $\Omega_{Ls_2} = -1.5511$.  We choose the {\em stopband edge} of the analog low pass filter as $\Omega_{Ls} = \mbox{min}(\vert \Omega_{Ls_1}\vert,\vert \Omega_{Ls_2}\vert) = 1.4653$.

\item {\em The Analog Lowpass Filter Transfer Function $H_{analog,LPF}(s_L)$:}  

\begin{equation}
\label{lpfinal}
H_{a,LP}(s_L) = \frac{0.3125}{s_L^4 + 1.1068s_L^3 + 1.6125s_L^2+0.9140s_L + 0.3366}
\end{equation}

\item {\em The Analog Bandpass Filter Transfer Function:}  

{\tiny
\begin{equation}
\label{bpfinal}
H_{a,BP}(s) = \frac{2.7776\times 10^{-5}s^4}{s^8+0.1055s^7+0.8589s^6+0.0676s^5+0.2735s^4+0.0143s^3+0.0383s^2+0.001s+0.002}
\end{equation}
}

\item {\em The Discrete-Time Filter Transfer Function:}

\begin{eqnarray}
H_{d,BP}(z) = G \frac{N(z)}{D(z)}
\end{eqnarray}
where $G = 2.7776 \times 10^{-5}$,

\begin{eqnarray}
N(z)=  1 - 4 z^{-2} + 6 z^{-4} - 4z^{-6} + z^{-8} 
\end{eqnarray}
and
\begin{eqnarray}
D(z) = 2.3609  -12.0002z^{-1} + 31.8772z^{-2}  -53.7495z^{-3}\nonumber \\+  62.8086z^{-4}
  -51.4634z^{-5}+   29.2231z^{-6}  -10.5329z^{-7} +   1.9842z^{-8}
\end{eqnarray}

\item {\em Filter Realization Using Direct Form II:} See attached page.

\item {\em Filter Realization Using Lattice Structure:} See attached page.

\item {\em FIR Filter Transfer Function Realization Using Kaiser Window:}

\begin{itemize}
\item The equivalent {\em lowpass} filter has {\em passband frequency $\omega_l = \frac{\omega_{p1} - \omega_{p2}}{2} = 0.025\pi$}.

\item The centre of the passband of the desired bandpass filter is $\omega_c = \frac{\omega_{p1} + \omega_{p2}}{2} = 0.275\pi$
\item For the given specifications, the {\em Kaiser} window reduces to the {\em rectangular} window of {\em length
$\geq 97$}.
\item The desired {\em lowpass filter impulse response}
\begin{eqnarray}
\label{firlpfinal}
h_{lp}(n) &=& \frac{\sin(\frac{n\pi}{40})}{n\pi} \hspace{2cm} -100 \leq n \leq 100 \nonumber \\
&=& 0, \hspace{4cm} \mbox{otherwise}
\end{eqnarray}

\item The desired {\em bandpass filter impulse response}
\begin{eqnarray}
\label{firbpfinal}
h_{bp}(n) &=& \frac{2\sin(\frac{n\pi}{40}) \cos(\frac{11n\pi}{40})}{n\pi} \hspace{1cm} -100 \leq n \leq 100 \nonumber \\
&=& 0, \hspace{4cm} \mbox{otherwise}
\end{eqnarray}

\item The {\em FIR filter transfer function coefficients} can be found in one of the attached pages.	
\end{itemize}

\end{enumerate}
\end{document}
